
\documentclass[letterpaper,12pt,]{article}

\date{April 10, 2015}

\usepackage{titling}

\setlength{\droptitle}{5in}   % This is your set screw

\usepackage[%
    left=1in,%
    right=1in,%
    top=1in,%
    bottom=1.0in,%
    paperheight=11in,%
    paperwidth=8.5in%
]{geometry}%
\usepackage{comment}

\usepackage{listings}
\usepackage{graphicx}
\usepackage{amsmath}
\usepackage[section]{placeins}
\usepackage[font=small,skip=-2pt]{caption}
\usepackage{subcaption}
\usepackage{hyperref}
\usepackage{booktabs}

\lstdefinestyle{mystyle}{
    %backgroundcolor=\color{backcolour},   
    %commentstyle=\color{codegreen},
    %keywordstyle=\color{magenta},
    %numberstyle=\tiny\color{codegray},
    %stringstyle=\color{codepurple},
    basicstyle=\footnotesize,
    breakatwhitespace=false,         
    breaklines=true,                 
    captionpos=b,                    
    keepspaces=true,                 
    numbers=left,                    
    numberstyle=\footnotesize,               
    stepnumber=1,
    numbersep=5pt,
    showspaces=false,                
    showstringspaces=false,
    showtabs=false,                  
    tabsize=2,
    frame=single
}
\lstset{frame=single}

\pagestyle{empty} % Remove page numbering
\linespread{1.5} % Line Spacing

\begin{document}

\begin{titlepage}

\newcommand{\HRule}{\rule{\linewidth}{0.5mm}} % Defines a new command for the horizontal lines, change thickness here

\center % Center everything on the page
 
%----------------------------------------------------------------------------------------
%	HEADING SECTIONS
%----------------------------------------------------------------------------------------


\textsc{\LARGE McGill University}\\[3.5cm]
\textsc{\Large Matrix Computations}\\[0.5cm] 
\textsc{\large COMP 540}\\[2.5cm]

%----------------------------------------------------------------------------------------
%	TITLE SECTION
%----------------------------------------------------------------------------------------

{ \huge \bfseries Assignment 6}\\[1.5cm] % Title of your document

\HRule \\[0.4cm]
%----------------------------------------------------------------------------------------
%	AUTHOR SECTION
%----------------------------------------------------------------------------------------

\begin{minipage}{0.4\textwidth}
\begin{flushleft} \large
\emph{Name:}\\
Doug \textsc{Shi-Dong} % Your name
\end{flushleft}
\end{minipage}
~
\begin{minipage}{0.4\textwidth}
\begin{flushright} \large
\emph{Student ID:} \\
260466662\\
\end{flushright}
\end{minipage}\\[4cm]

\vfill{}
{\large December 7, 2015}\\[2cm]

\end{titlepage}


\section*{Question 1}

\subsection*{Part 1}
Letting $a_{1}(t) = 1$ and $a_{2}(t) = t$, the single precision condition number of $A$ is $\kappa_2(A)=8.6798010\text{\sc{e}+}02$. The norm of the residual of the least-square solution by the QR factorization method is $\|b-Ax_{qr}\|_2 = 8.0615806\text{\sc{e}-}05$.

The condition number of the tranpose of $A$ with itself is $\kappa_{2}(A^{T}A) = 7.5613331\text{\sc{e}+}05$. The norm of the residual of the least-square solution by the normal equation is $\|b-Ax_{ne}\|_2 = 1.1265145\text{\sc{e}-}04$.

Using double precision for the $A$ and the $b$ matrix least allows to compare the previous solutions with the true solution. All the solutions have been compiled in Table \ref{tab:r1}.

\begin{table}[h]
\centering
\begin{tabular}{cccc} \toprule
    {} & {Single Precision QR} & {Single Precision NE} & {Double Precision QR}\\ \midrule
    $x_1$ &	1.0167466 & 1.028004 & 1.016737195131018\\
    $x_2$ &	0.9849972 & 0.9748591 & 0.985005503165425\\
\bottomrule
\end{tabular}
\caption{Results for $a_{1}(t) = 1$ and $a_{2}(t) = t$}
\label{tab:r1}
\end{table}

The single precision QR factorization is expected to give a better least-square solution than solving the normal equations. The condition number of $A^TA$ is three orders of magnitude bigger than the condition number of $A$. The QR factorization allows to retrieve up to 4 digits of accuracy, whereas the normal equation only retrieves 2 digits of the real solution.

\subsection*{Part 2}

Now, let $a_{1}(t) = 1$ and $a_{2}(t) = (t-1.11) \ast 100$. The single precision condition number of $A$ is $\kappa_2(A)=4.5301147$. The norm of the residual of the least-square solution by the QR factorization method is $\|b-Ax_{qr}\|_2 = 8.0615806\text{\sc{e}-}05$.

The condition number of the tranpose of $A$ with itself is $\kappa_{2}(A^{T}A) = 20.5219326$. The norm of the residual of the least-square solution by the normal equation is $\|b-Ax_{ne}\|_2 = 8.0527971\text{\sc{e}-}05$.

The single precision and double precision results have been compiled in Table \ref{tab:r2}.

\begin{table}[h]
\centering
\begin{tabular}{cccc} \toprule
    {} & {Single Precision QR} & {Single Precision NE} & {Double Precision QR}\\ \midrule
    $x_1$ &	2.1100941 & 2.1100929 & 2.110093286276619\\
    $x_2$ &	0.0098492 & 0.0098508 & 0.009850104587809\\
\bottomrule
\end{tabular}
\caption{Results for $a_{1}(t) = 1$ and $a_{2}(t) = (t-1.11) \ast 100$}
\label{tab:r2}
\end{table}

The change of basis has improved the condition number of both $A$ and $A^TA$ by a few orders of magnitude. Therefore, both $x_{qr}$ and $x_{ne}$ are much closer to the real solution $x_{qr}$. Note that the resulting $x_2$ in this new basis is just a multiple of the $x_2$ from the first basis.

\section*{Question 2}

Handwritten derivations are attached.

\section*{Question 3}

The singular values and vectors of $B$ are shown below.
\begin{equation*}
\lambda_B = \{-0.026463896961605, 2.404909433839755, -35.134394975254480\}
\end{equation*}
\begin{equation*}
U_B = 
\begin{bmatrix}
    0.6482 &  0.6737 &  0.3550 & -0.0000 &  0.0000\\
    0.6237 & -0.3900 & -0.3986 & -0.5140 &  0.1893\\
   -0.3149 &  0.0696 &  0.4427 & -0.5442 &  0.6355\\
    0.0060 &  0.2508 & -0.4869 &  0.4535 &  0.7031\\
    0.3028 & -0.5712 &  0.5310 &  0.4838 &  0.2569\\
\end{bmatrix}
\end{equation*}
\begin{equation*}
V_B = 
\begin{bmatrix}
    0.4183 &  0.8854 &  0.2026\\
   -0.8137 &  0.2662 &  0.5167\\
    0.4036 & -0.3810 &  0.8318\\
\end{bmatrix}
\end{equation*}
In order to test for stability, every entry of $B$ has been perturbed by $1\textsc{e-}05$. The error of the singular value and vectors are given by $\|S_{pert}-S_{B}\|_2=6.8766\textsc{e-}06$, $\|U_{pert}-U_{B}\|_2=1.9715\textsc{e-}05$, and $\|V_{pert}-V_{B}\|_2=1.0605\textsc{e-}07$. This test does not prove stability of the algorithm since only mathematical derivations can, but it does support that the algorithm is \emph{probably} numerically stable since a small error in the input lead to small error in the solution.

The SVD of A computed by MATLAB gives the following three singular values:
\begin{equation*}
\lambda_A = \{35.1272233335747, 2.46539669691652, 2.57621344955340\textsc{e-}16 \}
\end{equation*}

This shows that the $A$ matrix numerically has rank 3. However, it is easy to show that the columns of A are not linearly independent. Therefore, matrix $A$ mathematically has a rank of 2.

To test the validity of the Moore-Penrose matrix, the following properties are checked:
\begin{equation*}
\|AGA - A\| = 0 \quad 
\|GAG - G\| = 0 \quad 
\|(AG)^T - AG\| = 0 \quad 
\|(GA)^T - GA\| = 0
\end{equation*}

If we decide that the matrix 3 has full column rank, it is obvious that $\Sigma_1^{-1}$ blows up due to the small singular value. Therefore, none of the above test are satisfied and the resulting Moore-Penrose matrix is invalid.

Instead, if we say that $A$ has rank 2 instead of 3, the Moore-Penrose matrix given by $G = V_1 \Sigma_1^{-1} U_1^T$ satisfies the above tests. Moreover, the \texttt{pinv} MATLAB command confirms the results obtained.

The minimum 2-norm LS solution is given by $\hat{x} = Gb$. If rank 2 is chosen, $x_{LS2} = \[0.74, 0.20, -0.34]^T $ is the solution to the LS problem. However, if rank 3 is chosen, then 

\section*{Codes}

All codes are available on my GitHub:

\url{https://github.com/dougshidong/comp540/tree/master/a5}


\end{document}
